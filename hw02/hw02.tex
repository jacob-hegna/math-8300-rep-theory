\documentclass[reqno]{amsart}

\usepackage[utf8]{inputenc}
\usepackage{color}
\usepackage{extarrows}
\usepackage{tabu}
\usepackage{fullpage}
\usepackage{hyperref}
\usepackage{tikz-cd}
\usepackage{mathtools}
\usepackage{csquotes}
\usepackage{tikz}
\usepackage{faktor}
\usepackage{enumerate}
\usepackage{mathtools}
\usetikzlibrary{arrows}

\newcommand{\ds}[1]{\ensuremath{ \displaystyle{#1} }}

\title{Homework 02\\MATH 8300}
\author{Jacob Hegna}
\date{\today}

\newcommand{\Zz}{\mathbb{Z}}
\newcommand{\Gg}{\mathbb{G}}
\newcommand{\Hh}{\mathbb{H}}
\newcommand{\Top}{\mathbf{Top}}
\newcommand{\Sch}{\mathbf{Sch}}
\newcommand{\Aff}{\mathbf{Aff}}
\newcommand{\Ring}{\mathbf{Ring}}
\newcommand{\Set}{\mathbf{Set}}
\newcommand{\Sh}{\mathbf{Sh}}
\newcommand{\Ab}{\mathbf{Ab}}
\newcommand{\PAb}{\mathbf{PAb}}
\newcommand{\SmallCat}{\mathbf{SmallCat}}

\DeclareMathOperator{\Ob}{Ob}
\DeclareMathOperator{\Id}{Id}
\DeclareMathOperator{\End}{End}
\DeclareMathOperator{\Zar}{Zar}
\DeclareMathOperator{\Fun}{Fun}
\DeclareMathOperator{\Cov}{Cov}
\DeclareMathOperator{\Cl}{Cl}
\DeclareMathOperator{\St}{St}
\DeclareMathOperator{\link}{link}
\DeclareMathOperator{\Sub}{Sub}
\DeclareMathOperator{\Spec}{Spec}
\DeclareMathOperator{\Ann}{Ann}
\DeclareMathOperator{\Mod}{Mod}
\DeclareMathOperator{\mmod}{mod}
\DeclareMathOperator{\coker}{coker}
\DeclareMathOperator{\Comm}{Comm}
\DeclareMathOperator{\conv}{conv}
\DeclareMathOperator{\diag}{diag}
\DeclareMathOperator{\Ext}{Ext}
\DeclareMathOperator{\Tor}{Tor}
\DeclareMathOperator{\Gr}{Gr}
\DeclareMathOperator{\Hom}{Hom}
\DeclareMathOperator{\im}{im}

\newtheorem{theorem}{Theorem}[section]
\newtheorem{proposition}[theorem]{Proposition}
\newtheorem{lemma}[theorem]{Lemma}
\newtheorem{corollary}[theorem]{Corollary}
\theoremstyle{definition}
\newtheorem{definition}[theorem]{Definition}
\newtheorem{example}[theorem]{Example}
\newtheorem{remark}[theorem]{Remark}
\theoremstyle{remark}

\newcommand{\abs}[1] {
  \left| #1 \right|
}

\newcommand{\norm}[1] {
  \left| \abs{#1} \right|
}

\newcommand{\prob}[1] {
  \textbf{Problem #1.}
}

\newcommand\restr[2]{{% we make the whole thing an ordinary symbol
  \left.\kern-\nulldelimiterspace % automatically resize the bar with \right
  #1 % the function
  \vphantom{|} % pretend it's a little taller at normal size
  \right|_{#2} % this is the delimiter
  }}

\begin{document}
\maketitle

\prob{1} Suppose that $F : \mathcal{C} \to \mathcal{D}$ is an equivalence of
small categories.

\begin{enumerate}
  \item Show that, for all objects $x, y \in \Ob \mathcal{C}$, $F$ provides a
    bijection
    \[
      \Hom_\mathcal{C}(x, y) \to \Hom_\mathcal{D}(F(x), F(y)),
    \]
    so that $\End_\mathcal{C}(x) \cong \End_\mathcal{D}(F(x))$ as monoids.

    \begin{proof}
      By hypothesis, $F$ provides a set-theoretic map
      \[
        \Hom_\mathcal{C}(x, y) \to \Hom_\mathcal{D}(F(x), F(y)).
      \]
      In particular, as $F$ is an equivalence, we have a functor $G :
      \mathcal{D} \to \mathcal{C}$ which provides a map
      \[
        \Hom_\mathcal{D}(F(x), F(y)) \to \Hom_\mathcal{C}(GF(x), GF(y)).
      \]
      That $F$ and $G$ are components of an equivalence also implies there are
      natural transformations $\mu : \Id_\mathcal{C} \to GF$ and $\eta :
      \Id_\mathcal{D} \to FG$ which are isomorphisms in the functor category
      $\Fun(\mathcal{C}, \mathcal{D})$. In particular, this means $GF(x) \cong
      x$ for all $x \in \mathcal{C}$ and that $\Hom_\mathcal{C}(x, y) \cong
      \Hom_\mathcal{C}(GF(x), GF(y))$ (this is witnessed by $\mu$). Of course,
      we may say analogous things about objects in $\mathcal{D}$, the
      set-theoretic maps of $\Hom$-sets defined initially have two-sided
      inverses. This proves the original claim, and the only thing to verify is
      that these maps are compatible with the monoidal structure of
      $\End_\mathcal{C}(x)$ (but, of course, this follows from the compatibility
      conditions on the functor $F$ for $\mathcal{C}$-associativity).
    \end{proof}

  \item Show that $x \cong y$ in $\mathcal{C}$ if and only if $F(x) \cong F(y)$
    in $\mathcal{D}$.

    \begin{proof}
      If $x \xrightarrow{\phi} y$ is an isomorphism in $\mathcal{C}$, then
      $F(\phi)$ is an isomorphism in $\mathcal{D}$ (this is true for any
      functor). For the other direction, we consider an isomorphism $F(x)
      \xrightarrow{\theta} F(y)$ in $\mathcal{D}$. Using the same naming
      conventions as the previous problem, we have that $GF(x) \cong GF(y)$,
      however (as previously noted), $GF(x) \cong x$ due to the existence of the
      natural isomorphism $\mu$.
    \end{proof}

  \item Let $\mathcal{E}$ be a further category. Show that the functor
    categories $\Fun(\mathcal{C}, \mathcal{E})$ and $\Fun(\mathcal{D},
    \mathcal{E})$ are naturally equivalent.

    \begin{proof}
      Define a functor $A : \Fun(\mathcal{C}, \mathcal{E}) \to \Fun(\mathcal{D},
      \mathcal{E})$ by sending a functor $H$ to $G \circ H$. This defines a
      functor of functor categories by the observation that the composition of
      functors is a functor (say this sentence ten times fast). Similarly,
      define a functor $B$ via precomposition with $F$. We lift the natural
      transformations $\mu$, $\eta$ to the functor categories via a similar
      process. In particular, given two functors $I, J \in \Fun(\mathcal{C},
      \mathcal{E})$ with a natural transformation $\nu : I \to J$, we consider a
      lift of $\eta$:
      \[
        \begin{tikzcd}
          I \ar[r, "\nu"] \ar[d, "\hat{\eta}"] & J \ar[d, "\hat{\eta}"] \\
          BA(I) \ar[r] & BA(J)
        \end{tikzcd}
      \]
      where we note that $BA(I) = F G I$, and $\eta$ provides a map from $FG \to
      \Id_\mathcal{C}$. In particular, repeating this process to lift $\mu$ to
      $\hat{\mu}$ gives a natural isomorphism $A \cong B$, which finishes the
      proof.
    \end{proof}
\end{enumerate}

\prob{2} Let $C$ be a category and let $x, y \in \Ob \mathcal{C}$. Prove that if
$x \cong y$ then $\Hom(x, -)$ and $\Hom(y, -)$ are naturally isomorphic
functors.

\begin{proof}
  We provide two ways to see this. The first is to recall that the Yoneda
  embedding of a category into its presheaf category is fully faithful (there is
  a slight problem here that the presheaf category consists of the contravariant
  functors $\mathcal{C} \to \Set$, but the proof carries through in the
  covariant case similarly).

  The second, and much more fun, answer, is to note that this question is simply
  a special case of the final part of problem $(1)$, where we let $\mathcal{C}$
  be the $2$-category of small categories, then $x \cong y$ is the statement
  that ``$x$ is equivalent to $y$'' as categories, and the $\Hom$-sets become
  $\Fun$-sets.
\end{proof}

\prob{3} Let $F, G : \mathcal{C} \to \mathcal{D}$ be functors and $\eta : F \to
G$ be natural transformations.

\begin{enumerate}
  \item Show that if, for all $x \in \Ob \mathcal{C}$, the mapping $\eta_x :
    F(x) \to G(x)$ is an isomorphism in $\mathcal{D}$, then $\eta$ is a natural
    isomorphism.

    \begin{proof}
      The map $\eta_x$ has an inverse $\eta_x$. I claim that the collection of
      $\mu_x$ gives a natural transformation $\mu$ which inverts $\eta$. We
      see this by a careful examination of the following diagram:
      \[
        \begin{tikzcd}
          F(x) \ar[r] \ar[d, bend left, "\eta_x"] & F(y) \ar[d, bend left,
          "\eta_y"] \\
          G(x) \ar[r] \ar[u, bend left, "\mu_x"] & G(y) \ar[u, bend left,
          "\mu_y"]
        \end{tikzcd}
      \]
      That everything commutes follows from $\mu_x$ inverting $\eta_x$. In
      particular, we have a natural transformation, as we have provided a
      collection of necessary maps, and they satisfy the diagramatic criterion
      from requirement that $\mu_x$ is a (strict) inverse of $\eta_x$. That
      $\mu$ is a two sided inverse of $\eta$ is inherited from $\mu_x$ being a two
      sided inverse of $\eta_X$.
    \end{proof}

  \item Suppose that $F$ is an equivalence of categories and that $F$ is
    naturally isomorphic to $G$, so $F \simeq G$. Show that $G$ is an
    equivalence of categories.

    \begin{proof}
      Say that $F^{-1}$ is the functor which inverts $F$ up to a natural
      isomorphism $\eta$. Suppose $\mu$ witnesses $G \simeq F$. That is, we have
      an isomorphism $\mu_x : G(x) \cong F(x)$ for all $x$. Define the functor
      $G^{-1}$ on objects by $G^{-1}(x) = F^{-1}(\mu_x(x))$. It is compatible
      with morphisms because $\mu$ is a natural transformation. In particular,
      if we consider $G^{-1}G(x)$, we get a natural transformation to the
      identity functor from $\eta$.
    \end{proof}
\end{enumerate}

\prob{4} Let $\mathcal{C}$ be a small category. A \textit{self-equivalence} of
$\mathcal{C}$ is an equivalence of categories $F : \mathcal{C} \to \mathcal{C}$.
Show that the set of natural isomorphism classes of self equivalences of
$\mathcal{C}$ is a group, with multiplication induced by composition of
functors.

\begin{proof}
  We note that the second part of question ($3$) is what allows this definition
  to make sense at all.

  The identity element of the group is the equivalence class of the identity
  functor. We need to check that composition is well-defined, which will imply
  that the operation is associative (as on-the-nose functor composition is
  associative) and that there exist inverses (as, for any $F$ in the set, there
  exists an inverse up to natural isomorphism). So, we check this in two parts.
  Suppose $F_0 \simeq_{\mu} F_1$, we will show $F_0 G \simeq F_1 G$. We define a
  natural transformation $\eta$ where $\eta_x := \mu_{G(x)}$, which is a natural
  isomorphism with inverse given by the natural transformation taking $G^{-1}G$
  to the identity. Similarly, $G F_0 \simeq G F_1$ via $\tau_x := G(\mu_x)$. For
  clarity, we give the diagram for this second natural isomorphism:
  \[
    \begin{tikzcd}
      F_0(x) \ar[r] \ar[d, "\mu_x"] & F_0(y) \ar[d, "\mu_{y}"] \\
      F_1(x) \ar[r] & F_1(y)
    \end{tikzcd}
    \xLongrightarrow[]{G}
    \begin{tikzcd}
      GF_0(x) \ar[r] \ar[d, "G(\mu_x)"] & GF_0(y) \ar[d, "G(\mu_{y})"] \\
      GF_1(x) \ar[r] & GF_1(y)
    \end{tikzcd}
  \]
\end{proof}

\prob{5} Let $G$ be a group, which we regard as a category $\mathcal{G}$ with a
single object, and the elements of $G$ as morphisms. Let $F : \mathcal{G} \to
\mathcal{G}$ is a functor.

\begin{enumerate}
  \item Show that $F$ is naturally isomorphism to the identity functor iff the
    mapping $F : G \to G$ induced by $F$ is an inner automorphism.

    \begin{proof}
      Suppose $F$ is naturally isomorphic to the identity functor via $\eta$. In
      particular, $\eta_G \circ F(g) = g \circ \eta_G$ where $g$ is an element
      of $G$ regarded as a morphism. In particular, $F(g) = \eta_G^{-1} \circ g \circ
      \eta_g$. We check that $\eta_G$ must be multiplication by an element of
      $G$. In particular, natural transformation from $F$ to $\Id$ by definition
      associates to every object of the category a morphism from $F(G)$ to $G$,
      which in this case reduces to a choice of element of $G$.

      In the other direction, suppose that we have such an expression
      for $F$, $F(h) = g^{-1}hg$. So, we get a diagram
      \[
        \begin{tikzcd}
          G \ar[r, "h"] \ar[d, "g"] & G \ar[d, "g"] \\
          F(G) \ar[r, "F(h)"] & F(G)
        \end{tikzcd}
      \]
      Take $\eta_G = g$.
    \end{proof}

  \item Show that self equivalences of $\mathcal{G}$ are automorphisms of $G$.

    \begin{proof}
      Let $F$ be such a self equivalence. It maps $\End_\mathcal{G}(G)$ to
      itself, so it is a bijection, and by question ($1$), this is compatible
      with the monoid structure of $\End_\mathcal{G}(G)$. The monoid operation
      on this endomorphism set is the group operation of $G$ (by definition of
      the categorification of a group). This is all we needed to show.
    \end{proof}
\end{enumerate}

\end{document}
