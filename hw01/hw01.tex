\documentclass[reqno]{amsart}

\usepackage[utf8]{inputenc}
\usepackage{color}
\usepackage{tabu}
\usepackage{fullpage}
\usepackage{hyperref}
\usepackage{tikz-cd}
\usepackage{mathtools}
\usepackage{csquotes}
\usepackage{tikz}
\usepackage{faktor}
\usepackage{enumerate}
\usepackage{mathtools}
\usetikzlibrary{arrows}

\newcommand{\ds}[1]{\ensuremath{ \displaystyle{#1} }}

\title{Homework 01\\MATH 8300}
\author{Jacob Hegna}
\date{\today}

\newcommand{\Zz}{\mathbb{Z}}
\newcommand{\Gg}{\mathbb{G}}
\newcommand{\Hh}{\mathbb{H}}
\newcommand{\Top}{\mathbf{Top}}
\newcommand{\Sch}{\mathbf{Sch}}
\newcommand{\Aff}{\mathbf{Aff}}
\newcommand{\Ring}{\mathbf{Ring}}
\newcommand{\Set}{\mathbf{Set}}
\newcommand{\Sh}{\mathbf{Sh}}
\newcommand{\Ab}{\mathbf{Ab}}
\newcommand{\PAb}{\mathbf{PAb}}

\DeclareMathOperator{\Ob}{Ob}
\DeclareMathOperator{\Zar}{Zar}
\DeclareMathOperator{\Fun}{Fun}
\DeclareMathOperator{\Cov}{Cov}
\DeclareMathOperator{\Cl}{Cl}
\DeclareMathOperator{\St}{St}
\DeclareMathOperator{\link}{link}
\DeclareMathOperator{\Sub}{Sub}
\DeclareMathOperator{\Spec}{Spec}
\DeclareMathOperator{\Ann}{Ann}
\DeclareMathOperator{\Mod}{Mod}
\DeclareMathOperator{\mmod}{mod}
\DeclareMathOperator{\coker}{coker}
\DeclareMathOperator{\Comm}{Comm}
\DeclareMathOperator{\conv}{conv}
\DeclareMathOperator{\diag}{diag}
\DeclareMathOperator{\Ext}{Ext}
\DeclareMathOperator{\Tor}{Tor}
\DeclareMathOperator{\Gr}{Gr}
\DeclareMathOperator{\Hom}{Hom}
\DeclareMathOperator{\im}{im}

\newtheorem{theorem}{Theorem}[section]
\newtheorem{proposition}[theorem]{Proposition}
\newtheorem{lemma}[theorem]{Lemma}
\newtheorem{corollary}[theorem]{Corollary}
\theoremstyle{definition}
\newtheorem{definition}[theorem]{Definition}
\newtheorem{example}[theorem]{Example}
\newtheorem{remark}[theorem]{Remark}
\theoremstyle{remark}

\newcommand{\abs}[1] {
  \left| #1 \right|
}

\newcommand{\norm}[1] {
  \left| \abs{#1} \right|
}

\newcommand{\prob}[1] {
  \textbf{Problem #1.}
}

\newcommand\restr[2]{{% we make the whole thing an ordinary symbol
  \left.\kern-\nulldelimiterspace % automatically resize the bar with \right
  #1 % the function
  \vphantom{|} % pretend it's a little taller at normal size
  \right|_{#2} % this is the delimiter
  }}

\begin{document}
\maketitle

\prob{1}
\begin{enumerate}
  \item Describe all the isomorphism classes of representations of
    $\mathbb{C}[x]$ of dimension $1$. How many are there?
    \begin{proof}
      The modules $\mathbb{C}[x]/(x-z)$ for $z \in \mathbb{C}$ are all of
      dimension one (they are isomorphic to $\mathbb{C}$) and they are not
      mutually isomorphic for $(x-z) \neq (x-s)$, as the action of $x$ differs
      between them. Appealing to the structure theorem of modules over a PID, we
      see that these are indeed the only possible modules of dimension $1$.

      There are uncountably many such modules of distinct isomorphism classes
      when considered as $\mathbb{C}[x]$-modules. Base changing to $\mathbb{C}$
      collapses these to a single isomorphism class.
    \end{proof}

  \item Describe also the isomorphism classes of representations of
    $\mathbb{C}[X]$ of dimension $2$. Can they all be generated by a single
    element? If not, identify the representations that can be generated by a
    single element. Are any of these representations of dimension $2$ simple?

    \begin{proof}
      Once again, an appeal to the structure theorem says that any such module
      must be of the form
      \[
        \mathbb{C}[x] \bigoplus_i \frac{\mathbb{C}[x]}{I}.
      \]
      Indeed, such modules of dimension two are necessarily of the form
      $\mathbb{C}[x]/(x-a) \oplus \mathbb{C}[x]/(x-b)$ for $a, b \in \mathbb{C}$
      or $\mathbb{C}[x]/(x-a)^2$ for $a \in \mathbb{C}$. By the Chinese
      remainder theorem, $\mathbb{C}[x]/(x-a) \oplus \mathbb{C}[x]/(x-b) \cong
      \mathbb{C}/\left((x-a)(x-b)\right)$ for $a, b$ distinct. Thus, we reduce
      our classification to the modules which can be generated by a single
      element and those which cannot. The former class is of the form
      $\mathbb{C}[x]/(f)$ for $f$ degree $2$. The latter is of the form
      $\mathbb{C}[x]/(x-a) \oplus \mathbb{C}[x]/(x-a)$. To see this cannot be
      generated by a single element, suppose that there was such a generator
      $z$. Then, as the map from $\mathbb{C}[x]$ that multiplies by $z$ is
      surjective, which gives an isomorphism $\mathbb{C}[x]/(a)$ with the
      aforementioned module. However, this cannot be true, as the action of
      $x-a$ on the module $\mathbb{C}[x]/(x-a) \oplus \mathbb{C}[x]/(x-a)$ is
      trivial, and if that were true for $\mathbb{C}[x]/(z)$, then $z = x-a$
      up to units and the $\mathbb{C}[x]/(z)$ has dimension $1$ as a module,
      which does not equal the dimension of $\mathbb{C}[x]/(x-a) \oplus
      \mathbb{C}[x]/(x-a)$, which is $2$.
    \end{proof}
\end{enumerate}

\prob{2}
\begin{enumerate}
  \item Let $f \in \mathbb{Q}[x]$ be an irreducible polynomial. Show that every
    finitely generated module for the ring $A = \mathbb{Q}[x]/(f^r)$ is a direct
    sum of modules isomorphic to $V_s := \mathbb{Q}[x]/(f^s)$, where $1 \leq s
    \leq r$. Show that $A$ has only one simple module up to isomorphism. When $r
    = 5$, calculate $\dim \Hom_A(V_2, V_4)$ and $\dim \Hom(V_4, V_2)$.

    \begin{proof}
      Of course, that every finitely generated module over $A$ is a direct sum
      of $V_s$ comes immediately from the structure theorem, as we have that the
      module must be of the form
      \[
        \frac{\mathbb{Q}[x]}{f^r} \bigoplus_i
        \frac{\mathbb{Q}[x]/(f^r)}{(f^s)/(f^r)},
      \]
      which is isomorphic to $\bigoplus_i V_{s_i}$ for $s_i \leq r$.

      That $A$ has only one simple module comes from a more general fact that
      any (commutative, Noetherian) local ring $(R, m)$ has a unique simple
      module isomorphic to $R/m$---to demonstrate this fact, consider the map $A
      \to M$ given by fixing a nonzero element $a$ of the simple module $M$ and
      taking the map to be multiplication by $a$. Surjectivity is implied by
      simplicity of $M$, and thus $M$ is only of length $1$ if the kernel of
      this map is $m$.

      Let $r=5$. $\Hom_A(V_2, V_4)$ is the collection of maps $A \to V_4$ which
      annihilate $f^2$. This only occurs if $1$ is mapped into $(f^2)$. The
      length of $\Hom_A(V_2, V_4)$ is given by considering that the submodules
      are classified by which ideal $1$ is sent to in the underlying map $A \to
      V_4$, of which there are three choices, $0 \subset (f^3) \subset (f^2)$.
      This gives the desired length: $2$. The module $\Hom_A(V_4, V_2)$ is
      similarly classified, and $1$ may be mapped into either of the ideals $0
      \subset (f) \subset V_2$, which gives a length of $2$ as well.
    \end{proof}
\end{enumerate}

\prob{2} Show that $\mathbb{Q}[x]/((x-1)^5) \cong \mathbb{Q}[x]/((x-2)^5)$ as
algebras.

\begin{proof}
  As $\mathbb{Q}$-algebras, they are each isomorphic to
  $\mathbb{Q}[\epsilon]/(\epsilon^5)$ under the map sending $\epsilon$ to $x-1$
  and $x-2$, respectively.

  They are not isomorphic as $\mathbb{Q}[x]$-algebras, however, as there is no
  surjective $\mathbb{Q}[x]$-linear arrow making the following diagram commute:
  \[
    \begin{tikzcd}
      &\mathbb{Q}[x] \ar[dr] \ar[dl] & \\
      \frac{\mathbb{Q}[x]}{(x-1)^5} & & \frac{\mathbb{Q}[x]}{(x-2)^5}
    \end{tikzcd}
  \]
  This can be seen by the fact that $x-2$ is sent to a unit in the left map and
  sent to a nonunit in the right map. To prove that $x-2$ is a unit in
  $\mathbb{Q}[x]/((x-1)^5)$, it is sufficient to observe that as $(x-1)^5$ and
  $(x-2)$ are coprime in $\mathbb{Q}[x]$, there are polynomials $p(x)$, $q(x)$
  so that
  \[
    p(x)(x-2) + q(x)(x-1)^5 = 1,
  \]
  and so $p(x)$ is the multiplicative inverse for $(x-2)$ in
  $\mathbb{Q}[x]/((x-1)^5)$.

  Perhaps easier to see is that each of these algebras are local, and their
  residue fields are not isomorphic as $\mathbb{Q}$-algebras as the action of
  $x$ are different in either.
\end{proof}

\prob{3} Let $A$ be a ring and let $V$ be an $A$-module.

\begin{enumerate}
  \item Show that $V$ is simple if and only if for all nonzero $x \in V$, $x$
    generates $V$.

    \begin{proof}
      Suppose there is a nonzero element $x$ which does not generate $V$. What
      it generates is a nonzero submodule strictly contained in $V$ which is
      absurd as $V$ is simple. In the other direction, there can be so nonzero
      submodules of $V$ as all of their elements generate the entirity of $V$.
    \end{proof}

  \item Show that $V$ is simple if and only if $V$ is isomorphic to $A/I$ for
    some maximal left ideal $I$.

    \begin{proof}
      Fix a nonzero $v \in V$ and consider the map $A \to V$ which sends $a$ to
      $am$. This is surjective by the prior question. The kernel of this map
      must be maximal, otherwise the maximal ideal containing the kernel would
      yield a nonzero strict submodule of $V$ as a quotient of $A$. Thus $V
      \cong A/I$ for $I$ maximal.
    \end{proof}

  \item Show that if $A$ is a finite dimensional algebra over a field then every
    simple $A$-module is a composition factor of the free rank $1$ module
    $\prescript{}{A}{A}$, and hence that a finite dimensional algebra only has
    finitely many isomorphism classes of simple modules.

    \begin{proof}
      Fix a simple module $M \cong A/m$ for $m$ a maximal left ideal. We write a
      composition series for $\prescript{}{A}{A}$ which begins with the
      inclusion $m \to A$ (we will drop the left-module subscript notation, e.g.
      $\prescript{}{A}{A}$, as from now on everything in sight is acted on the
      left by $A$), this begins a composition series as $A/m \cong M$ is simple.
      We may extend this to $0$ by observing that ideals are vector subspaces of
      $A$, and thus we may inductively choose a maximal subideal of each element
      of the composition series, which terminates as each step decreases the
      dimension by $1$. By Jordan-Hölder, there are only finitely many such
      (isomorphism classes of) simple modules, as the composition series is
      unique up to reordering of the quotient modules.
    \end{proof}
\end{enumerate}

\prob{4} Let $K$ be a field, and let $Q_2 = y \bullet \xleftarrow{\beta}
\bullet x$ be the quiver in the notes with representations $S_x = 0
\xleftarrow{0} K$, $S_y = K \xleftarrow{0} 0$, and $V = K \xleftarrow{1} K$.

\begin{enumerate}
  \item Compute $\dim \Hom_{K(F(Q_2))}(S_x, V)$, $\dim \Hom_{K(F(Q_2))}(V, S_x)$
    and $\dim \Hom_{K(F(Q_2))}(V, V)$.
    \begin{proof}
      We consider the following diagram
      \[
        \begin{tikzcd}
          K \ar[d, "0"] \ar[r, "\alpha"] & K \ar[d, "1"] \\
          0 \ar[r, "\beta"] & K
        \end{tikzcd}
      \]
      The only choice for $\beta$ is $0$. To make things commute, we require
      $\alpha = 0$ as well, which implies the dimension of the $\Hom$-module is
      $0$. Similarly, we may consider
      \[
        \begin{tikzcd}
          K \ar[d, "1"] \ar[r, "\alpha"] & K \ar[d, "0"] \\
          K \ar[r, "\beta"] & 0
        \end{tikzcd}
      \]
      $\beta$ must be $0$, and $\alpha$ can be anything. Thus, the dimension of
      $\Hom(V, S_x)$ is $1$. Finally, we may consider
      \[
        \begin{tikzcd}
          K \ar[d, "1"] \ar[r, "\alpha"] & K \ar[d, "1"] \\
          K \ar[r, "\beta"] & K
        \end{tikzcd}
      \]
      For which we require $\alpha=\beta$ for commutativity, which yields a
      dimension of $1$ for $\Hom(V, V)$.
    \end{proof}

  \item Determine whether or not the path algebra $K(F(Q_2))$ is isomorphic to
    either $K[x]/(x^2)$ or $K[x]/(x^3)$.

    \begin{proof}

    \end{proof}
\end{enumerate}

\prob{5} Show that the path algebras of the two quivers $Q_1 = \bullet \to
\bullet \xleftarrow{} \bullet$ and $Q_2 = \bullet \xleftarrow{} \bullet \to
\bullet$ over $R$ are isomorphic to the algebras of $3 \times 3$ matrices over
$R$ of the form
\[
  R_1 = \begin{bmatrix}
    a & b & 0 \\
    0 & c & 0 \\
    0 & d & e
  \end{bmatrix}
  \quad\text{and}\quad
  R_2 = \begin{bmatrix}
    a & 0 & 0 \\
    b & c & d \\
    0 & 0 & e
  \end{bmatrix}
\]

\begin{proof}
  I claim that $K(F(Q_1)) = R_2$ and $K(F(Q_2)) = R_1$. To see this, we write
  out the multiplication tables. Write $Q_1 = x \xrightarrow{\alpha} y
  \xleftarrow{\beta} z$. The multiplication tables are:
  \[
    \begin{tabu}{c|ccccc}
      K(F(Q_1)) & 1_x & 1_y & 1_z & \alpha & \beta \\
      \hline
      1_x & 1_x & 0 & 0 & 0 & 0 \\
      1_y & 0 & 1_y & 0 & \alpha & \beta \\
      1_z & 0 & 0 & 1_z & 0 & 0 \\
      \alpha & \alpha & 0 & 0 & 0 & 0 \\
      \beta & 0 & 0 & \beta & 0 & 0
    \end{tabu}
    \quad\text{and}\quad
    \begin{tabu}{c|ccccc}
      R_2 & a & c & e & b & d \\
      \hline
      a & a & 0 & 0 & 0 & 0 \\
      c & 0 & c & 0 & b & d \\
      e & 0 & 0 & e & 0 & 0 \\
      b & b & 0 & 0 & 0 & 0 \\
      d & 0 & 0 & d & 0 & 0
    \end{tabu},
  \]
  where we write $i \in \{a, b, c, d, e\}$ for the matrix with $1$ in the $i$th
  place, where $i$ refers to the coordinate of $R_1$ indicated in the statement
  of the question. These algebras are necessarily now the same, as mapping
  obvious elements together, it is immediate that addition is preserved at $1 =
  1_x + 1_y + 1_z \mapsto 1 = a + c + e$.

  It is not too hard to see that transposing this multiplication table
  corresponds to transposing $R_1$ to $R_2$ and similarly reversing the
  direction of $\alpha$, $\beta$ (as transposing the multiplication table
  simply conjugates the endpoints of the path). However, for clarity, we write
  the multiplication tables below:
  \[
    \begin{tabu}{c|ccccc}
      K(F(Q_2)) & 1_x & 1_y & 1_z & \alpha & \beta \\
      \hline
      1_x & 1_x & 0 & 0 & \alpha & 0 \\
      1_y & 0 & 1_y & 0 & 0 & 0 \\
      1_z & 0 & 0 & 1_z & 0 & \beta \\
      \alpha & 0 & \alpha & 0 & 0 & 0 \\
      \beta & 0 & \beta & 0 & 0 & 0
    \end{tabu}
    \quad\text{and}\quad
    \begin{tabu}{c|ccccc}
      R_1 & a & c & e & b & d \\
      \hline
      a & a & 0 & 0 & b & 0 \\
      c & 0 & c & 0 & 0 & 0 \\
      e & 0 & 0 & e & 0 & d \\
      b & 0 & b & 0 & 0 & 0 \\
      d & 0 & d & 0 & 0 & 0
    \end{tabu}.
  \]
\end{proof}

\prob{6} Let $k$ be a field. Show that the space of column vectors $k^n$ is a
simple module for $M_n(k)$. Show that, as a left module, $M_n(k)$ is the direct
sum of $n$ modules each isomorphic to $k^n$. Show that, up to isomorphism,
$M_n(k)$ has only one simple module.

\begin{proof}
  Let $v \in k^n$ be a nonzero column vector. We consider $k^n$ a vector space
  with standard basis. As $v$ is nonzero, it has a nonzero coordinate $i$.
  Indeed, as $k$ is a field, we may choose a matrix which maps $v$ to the $i$th
  basis vector. There exists a linear transformation which sends this basis
  vector to any other basis vector in $k^n$ of our liking, which is expressed as
  an element of $M_n(k)$. This implies that $v$ generates all of $k^n$ and thus
  $k^n$ is simple.

  Let $N_i \subset M_n(k)$ refer to the submodule of matrices which are zero
  outside of the $i$th column. This is a submodule as it is a subset which
  inherits the addition and action of $M_n(k)$, and it is closed under this
  action by the definition of matrix multiplication. There is an obvious
  isomorphism of $N_i \cong k^n$ by identifying $k^n$ with the $i$th column of
  any given matrix. Of course, we have a surjection $M_n(k) \to \bigoplus_i^n
  N_i$ which projects the $i$th column to $N_i$, and its kernel is zero by
  definition.

  $M_n(k)$ has only one simple module (up to isomorphism) as the composition
  series
  \[
    0 \subset N_0 \subset N_0 \oplus N_1 \subset \cdots \subset
    \bigoplus_{i=0}^{n-1} N_i = M_n(k)
  \]
  has composition factors $N_i$ and the $i$th place, which is simple as they
  agree with $k^n$. By Jordan-Hölder, we conclude.
\end{proof}

\prob{7} Let $C_n$ denote the category with $n$ objects labeled $a_1, \dots,
a_n$ and where there is a unique homomorphism $a_i \to a_j$ for every ordered
pair of numbers $(i, j)$. Show that the category algebra $RC_n$ is isomorphic to
the algebra of $n \times n$ matrices $M_n(R)$.

\begin{proof}
  This is simply unpacking the definitions. The algebra $RC_n$ has, as a
  generating set, the arrows $a_i \to a_j =: \beta_{ij}$. The multiplication is
  defined to be:
  \[
    X = \left(\sum_{i \leq j} b_{ij} \beta_{ij}\right) \cdot \left(\sum_{i \leq
        j} c_{ij} \beta_{ij}\right),
  \]
  where the coefficient of $\beta_{ij}$ in $X$ is exactly the sum $\sum_{k=1}^n
  b_{ik} c_{kj}$, because the product is zero if the morphisms are not
  composible and the $R$-inherited product otherwise. If we consider the map of
  sets $RC_n \to M_n(R)$ sending $\beta_{ij}$ to the unit in the $ij$th
  coordinate of $M_n(R)$ then extending $R$-linearly, we find this is precisely
  the definition of matrix multiplication.
\end{proof}

\prob{8} Let $x$ be an object of a finite category $\mathcal{C}$.

\begin{enumerate}
  \item Show that the subset $R \mathcal{C} \cdot 1_x$ of the category algebra
    $R \mathcal{C}$ is the span of the morphisms whose domain is $x$, and that
    $1_x \cdot R\mathcal{C}$ is the span of the morphisms whose codomain is $x$.

    \begin{proof}
      Of course, this is just the definition of the multiplication in the
      category algebra. For any $r \beta \in R \mathcal{C}$, we have that $(r
      \beta) \cdot 1_x$ is $0$ if $o(\beta) \neq x$ and $r \beta$ otherwise.
      This is the definition of the desired span. For the other direction, the
      argument is the same, but we consider $t(\beta)$ as the multiplication is
      on the other side.
    \end{proof}

  \item Show that $R \mathcal{C} = \bigoplus_{x \in \Ob{\mathcal{C}}} R
    \mathcal{C} \cdot 1_x$ as left $R\mathcal{C}$-modules.

    \begin{proof}
      Define a map $\phi : R\mathcal{C} \to \bigoplus R \mathcal{C} \cdot 1_x$
      by projecting a formal linear sum of morphisms to the direct summand of
      its domain. By part ($1$), this is surjective. The only element mapped to
      the identity is $\sum_{x\in\Ob{\mathcal{C}}} 1_x$, so it is a bijection.
      The projection, of course, preserves the addition. So, we only have to
      check that the map is stable under the action of the ring. Left
      multiplication by $R \mathcal{C}$ preserves the domain of a morphism, so
      the map is a bijective $R\mathcal{C}$-module homomorphism.
    \end{proof}

  \item Let $R \Hom_\mathcal{C}(x, -)$ denote the functor $C \to R-\mmod$ that
    sends an object $y$ to the free $R$-module with the set of homomorphisms
    $\Hom_\mathcal{C}(x, y)$ as a basis. Under the correspondence between
    representations of $\mathcal{C}$ over $R$ and $R\mathcal{C}$-modules, show
    that the functor $R \Hom(x, -)$ corresponds to the left
    $R\mathcal{C}$-module $R\mathcal{C}\cdot 1_x$ and that $R \Hom(-, x)$
    corresponds to the right $R\mathcal{C}$-module $1_x \cdot R \mathcal{C}$.

    \begin{proof}
      Under the Mitchell-correspondence, the module formed from the
      representation (functor) given by $\Hom(x, -)$ is precisely the direct sum
      over the objects $x$ in $\mathcal{C}$, e.g. $\bigoplus_{y \in
        \Ob{\mathcal{C}}} R \Hom(x, y)$. This is, by definition, the $R$-linear
      span of all morphisms whose domain is $x$. By ($1$), this is $R
      \mathcal{C} \cdot 1_x$. The only thing to check is that the action of
      $R\mathcal{C}$ agrees on each. Of course, by definition, $R\Hom(x, y)$ is
      acted on by the basis elements of $R\mathcal{C}$ by composition, which is
      the same action of $R \mathcal{C} \cdot 1_x$. The same procedure shows
      that $R \Hom(-, x)$ corresponds to $1_x \cdot R \mathcal{C}$.
    \end{proof}
\end{enumerate}

\end{document}
